
Hola Elisa! :D

Te dejo por aquí una notas que te serán de utilidad, cualquier duda o cambio que veas que es necesario házmelo saber.

   - Asegurarse que el documento root de la compilación es main.tex
   
   - Asegurarse que el tipo de compilador XeLaTeX. Si lo vas a compilar en local, asegúrate de que en la secuencia de compilación de XelaTeX le indicas el parámetro --shell-escape.
   
   - La estructura del proyecto es la siguiente:   

           [+] archivos (Aquí irán figuras, código, etc)
           [+] body
            |
            |---[+] anexos (Anexos, lista de requerimientos, presupuesto, y acrónimos)
            |---[+] bibliografía (Biblio n.n )
            |---[+] capitulos (Capítulos de la memoria, resúmenes, agradecimientos)
            |           
           [+] include (Portada del otro proyecto, ficheros de configuración)
           

Solo tengo una duda respecto a la lista de acrónimos, por lo que vi la plantilla de J.Macias la ponía antes de los capítulos, en cambio, en esta universidad la ponen al final.. ¿Qué es mejor?

