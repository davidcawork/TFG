\usepackage[utf8]{inputenc}
\usepackage[spanish]{babel}
\usepackage{ifthen}
\newcommand{\colorspaceused}{rgb}
\ifthenelse{\equal{\colorspaceused}{rgb}}
{
  \PassOptionsToPackage{rgb}{xcolor}% NB: put this *before* \usepackage{pst-all}
}
{
  \PassOptionsToPackage{cmyk}{xcolor}% NB: put this *before* \usepackage{pst-all}
}
\usepackage{ae}               
\usepackage{lmodern}      
\usepackage{emptypage}
\usepackage{listings}
\usepackage{fancyhdr}
\usepackage{xcolor}
\definecolor{pantone293}{RGB}{35,91,168}
\definecolor{headingPortadaTFG}{RGB}{152,118,52}
\definecolor{headingPortadaTFM}{RGB}{0,90,170}
\definecolor{textoHeadingPortadaTFM}{RGB}{208,205,102}
\definecolor{textoHeadingPortadaTFG}{RGB}{208,205,102}
\definecolor{gray97}{gray}{.97}
\definecolor{gray75}{gray}{.75}
\definecolor{gray45}{gray}{.45}
\usepackage{tikz}
\usepackage{geometry}
\geometry{verbose,a4paper,tmargin=2.5cm,bmargin=2.5cm,lmargin=2.5cm,rmargin=2.5cm}
\usepackage[              
bookmarks=true,
bookmarksnumbered=true,        
hypertexnames=false,              
breaklinks=true,                  
linktoc=all,
colorlinks=true,
linkcolor=blue,    
citecolor=green,
urlcolor=blue,     
pdfborder={0 0 112.0},              
hyperfootnotes=false,
]{hyperref}                        

% Para numerar las \subsubsection
\setcounter{secnumdepth}{5}
% para hacer que las \subsubsection aparezcan en el indice
\setcounter{tocdepth}{5}
% \setcounter{lofdepth}{2}
\setcounter{table}{1}
\setcounter{figure}{1}
\setcounter{secnumdepth}{4}


\setlength{\parskip}{1ex plus 0.5ex minus 0.2ex}


\usepackage{multirow}

\usepackage{setspace}
% \renewcommand{\baselinestretch}{10}
\newcommand{\mycaptiontable}[1]{
  \begin{spacing}{0.6}
    % \vspace{0.5cm}
    \begin{quote}
      % \begin{center}
      {{Table} \thechapter.\arabic{table}: #1}
      % \end{center}
    \end{quote}
    % \vspace{1cm}
  \end{spacing}
  \stepcounter{table}
}

\newcommand{\mycaptionfigure}[1]{
  % \vspace{0.5cm}
  \begin{spacing}{0.6}
    \begin{quote}
      % \begin{center}
      {{Figure} \thechapter.\arabic{figure}: #1}
      % \end{center}
    \end{quote}
    % \vspace{1cm}
  \end{spacing}
  \stepcounter{figure}
}

\usepackage{amsmath}

\usepackage{courier}

% ***************************************************************************
% ***************************************************************************
% ***************************************************************************
\usepackage{multirow}
%\usepackage{rotating}
\usepackage{setspace, amssymb, amsmath, epsfig, multirow, colortbl, tabularx}
\usepackage[acronym,shortcuts,nomain,hyperfirst=false]{glossaries}
\newcommand{\clearemptydoublepage}{\newpage{\pagestyle{empty}\cleardoublepage}}
\pagestyle{fancy}
\providecommand\phantomsection{}
\onehalfspacing
\sloppy  %better line breaks
\renewcommand{\chaptermark}[1]{\markboth{\chaptername\ \thechapter.\ #1}{}}
\renewcommand{\sectionmark}[1]{\markright{\thesection\ #1}{}}

%%%%%%%%%%%%%%%%%%%%%%%%%%%%%%%%%%%%%%%%%%%%%%%%%%%%%%%%%%%%%%%%%%%%%%%%%%% 
% BEGIN Fancy headers stuff
\fancyhf{}

\fancyhead[LE,RO]{\bfseries\thepage}
\fancyhead[LO]{\bfseries\rightmark}
\fancyhead[RE]{\bfseries\leftmark}

\makeatletter
\renewcommand{\chaptermark}[1]{\markboth{\@chapapp \ \thechapter . \ #1}{}}
\renewcommand{\sectionmark}[1]{\markright{\thesection \ \ #1}}
\makeatother

\renewcommand{\headrulewidth}{0.5pt}
\renewcommand{\footrulewidth}{0pt}
\addtolength{\headheight}{3.5pt}
\fancypagestyle{plain}{\fancyhead{}\renewcommand{\headrulewidth}{0pt}}
\fancypagestyle{myplain}
{
  \fancyhf{}
  \renewcommand\headrulewidth{0pt}
  \renewcommand\footrulewidth{0pt}
  \fancyfoot[C]{\thepage}
}

%%%%%%%%%%%%%%%%%%%%%%%%%%%%%%%%%%%%%%%%%%%%%%%%%%%%%%%%%%%%%%%%%%%%%%%%%%%   
%   This is to set background images (in our case to set background image
%   in TFMs front and back pages)
%   If you want to set this background, use \BgThispage in the
%   corresponding pages
\usepackage[pages=some]{sty/background}

\ifthenelse{\equal{\colorspaceused}{rgb}}
{
  \backgroundsetup{ scale=1, angle=0, opacity=.1, color=pink,
    contents={\includegraphics[width=.7\paperwidth]{logos/logoEPS-UAH.jpg}}, vshift=-50pt,  hshift=0pt }
}
{
  \backgroundsetup{ scale=1, angle=0, opacity=.1, color=pink,
    contents={\includegraphics[width=.7\paperwidth]{logos/logoEPS-UAH-cmyk.jpg}}, vshift=-50pt,  hshift=0pt }
}


% This is to allow do a clearpage and let the next one to be placed in
% even pages (to set a backpage for example)
\makeatletter
\newcommand*{\cleartoleftpage}{%
  \clearpage
  \if@twoside
  \ifodd\c@page
  \hbox{}\newpage
  \if@twocolumn
  \hbox{}\newpage
  \fi
  \fi
  \fi
}
\makeatother
\usepackage{float}
\floatstyle{plaintop} % optionally change the style of the new float
\newfloat{codefloat}{H}{cod}[chapter]

\lstdefinestyle{console}
{
  basicstyle=\scriptsize\bf\ttfamily,
  backgroundcolor=\color{gray75},
}

\lstdefinestyle{Cbluebox}
{
  language=C,
  frame=shadowbox, 
  rulesepcolor=\color{blue}
}

\lstdefinestyle{Cnice}
{
  language=C,
  frame=Ltb,
  framerule=0pt,
  tabsize=2,
  aboveskip=0.5cm,
  framextopmargin=3pt,
  framexbottommargin=3pt,
  framexleftmargin=0.4cm,
  framesep=0pt,
  rulesep=.4pt,
  backgroundcolor=\color{gray97},
  rulesepcolor=\color{black},
  % 
  stringstyle=\ttfamily,
  showstringspaces = false,
  % basicstyle=\small\ttfamily,
  basicstyle=\footnotesize\ttfamily,
  commentstyle=\color{gray45},
  keywordstyle=\bfseries,
  % 
  numbers=left,
  numbersep=15pt,
  numberstyle=\tiny,
  numberfirstline = false,
  breaklines=true,
}	

\lstdefinestyle{CppExample}
{
  language=C++,
  frame=trbl,
  tabsize=2,
  commentstyle=\textit,
  stringstyle=\ttfamily, 
  basicstyle=\small,
}	

% This one from http://en.wikibooks.org/wiki/LaTeX/Source_Code_Listings
\lstdefinestyle{Ccolor}
{
  belowcaptionskip=1\baselineskip,
  breaklines=true,
  frame=L,
  xleftmargin=\parindent,
  language=C,
  showstringspaces=false,
  basicstyle=\footnotesize\ttfamily,
  keywordstyle=\bfseries\color{green!40!black},
  commentstyle=\itshape\color{purple!40!black},
  identifierstyle=\color{blue},
  stringstyle=\color{orange},
}

% From http://tex.stackexchange.com/questions/46953/unix-command-highlighting-latex
\lstdefinestyle{BashInputStyle}{
  language=bash,
  basicstyle=\small\sffamily,
  numbers=left,
  numberstyle=\tiny,
  numbersep=3pt,
  frame=tb, 
  showspaces=false, 
  showtabs=false,
  showstringspaces=false,
  columns=fullflexible,
  backgroundcolor=\color{gray97},
  % backgroundcolor=\color{yellow!20},
  linewidth=0.9\linewidth,
  xleftmargin=0.05\linewidth
}


% To set side-captions in figures
\usepackage{sidecap}


\def\texis{\TeX \raise.15em\hbox{\textsc{i}}S}   
\newenvironment{FraseCelebre}% Definición del entorno de FraseCelebre
{\begin{list}{}{%
      \setlength{\leftmargin}{0.5\textwidth}% Desplazamos el inicio de
      % los párrafos a la derecha la mitad
      % de la anchura de la línea de texto.
      % Puede que quieras cambiar esto
      % por otra cantidad como '5cm'.
      \setlength{\parsep}{0cm}% La separación entre párrafos de la
      % frase o de la fuente es normal, sin
      % espacio extra.
      \addtolength{\topsep}{0.5cm}% Aumentamos un poco la separación
      % entre la parte de la fase célebre
      % y los párrafos de alrededor
    }
  }
  {\unskip \end{list}}

\newenvironment{Frase}%
{\item \begin{flushright}\small\em}%
  {\end{flushright}}

\newenvironment{Fuente}%
{\item \begin{flushright}\small}%
  {\end{flushright}}

\newenvironment{bottomparagraph}{\par\vspace*{\fill}}{\clearpage}
\usepackage[vlined,algochapter]{algorithm2e}
\providecommand{\DontPrintSemicolon}{\dontprintsemicolon}
\providecommand{\SetAlgoLined}{\SetLine}
\usepackage{fix-cm}
\usepackage{graphicx}                                                              
\newcommand{\myreferencespath}{}
\providecommand{\DIFadd}[1]{{\protect\color{blue}\textbf{#1}}}
\providecommand{\DIFdel}[1]{{\protect\color{red}\sout{#1}}}                     


